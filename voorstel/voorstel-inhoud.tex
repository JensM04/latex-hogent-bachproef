%---------- Inleiding ---------------------------------------------------------
\section{Inleiding}%
\label{sec:inleiding}

Waarover zal je bachelorproef gaan? Introduceer het thema en zorg dat volgende zaken zeker duidelijk aanwezig zijn:

\begin{itemize}
  \item kaderen thema
  \item de doelgroep
  \item de probleemstelling en (centrale) onderzoeksvraag
  \item de onderzoeksdoelstelling
\end{itemize}

Denk er aan: een typische bachelorproef is \textit{toegepast onderzoek}, wat betekent dat je start vanuit een concrete probleemsituatie in bedrijfscontext, een \textbf{casus}. Het is belangrijk om je onderwerp goed af te bakenen: je gaat voor die \textit{ene specifieke probleemsituatie} op zoek naar een goede oplossing, op basis van de huidige kennis in het vakgebied.

De doelgroep moet ook concreet en duidelijk zijn, dus geen algemene of vaag gedefinieerde groepen zoals \emph{bedrijven}, \emph{developers}, \emph{Vlamingen}, enz. Je richt je in elk geval op it-professionals, een bachelorproef is geen populariserende tekst. Eén specifiek bedrijf (die te maken hebben met een concrete probleemsituatie) is dus beter dan \emph{bedrijven} in het algemeen.

Formuleer duidelijk de onderzoeksvraag! De begeleiders lezen nog steeds te veel voorstellen waarin we geen onderzoeksvraag terugvinden.

Schrijf ook iets over de doelstelling. Wat zie je als het concrete eindresultaat van je onderzoek, naast de uitgeschreven scriptie? Is het een proof-of-concept, een rapport met aanbevelingen, \ldots Met welk eindresultaat kan je je bachelorproef als een succes beschouwen?

%---------- Stand van zaken ---------------------------------------------------

\section{Literatuurstudie}%
\label{sec:literatuurstudie}

% Voor literatuurverwijzingen zijn er twee belangrijke commando's:
% \autocite{KEY} => (Auteur, jaartal) Gebruik dit als de naam van de auteur
%   geen onderdeel is van de zin.
% \textcite{KEY} => Auteur (jaartal)  Gebruik dit als de auteursnaam wel een
%   functie heeft in de zin (bv. ``Uit onderzoek door Doll & Hill (1954) bleek
%   ...'' Als je informatie over bronnen verzamelt in JabRef, zorg er dan voor dat alle nodige info aanwezig is om de bron terug te vinden (zoals uitvoerig besproken in de lessen Research Methods).)

Wanneer men een bezoek brengt aan het kerkhof verwacht je graag geen stress of vermoeidheid. Het vinden van een dierbaar persoon kan vaak langer duren dan verwacht bij een begraafplaats. Dit kan bijvoorbeeld door een groot aantal bezoekers, of een zeer grote of complex gelegen begraafplaats. Het fysiek zoeken kan zeer tijdrovend zijn en is niet altijd even efficiënt. Dit resulteert in een stressvol of ongemakkelijk gevoel, wat niet prettig is bij een bezoek van een geliefde. \Autocite{Haekkilae2022} 

Hierop kan digitalizatie een oplossing bieden. In het dagelijks leven en zeker in 2025, zie je overal digitalizatie. Maar hoe zit dat op kerkhoven? Kan dit op een eficiente manier toegepast worden zonder het de bezoekers te moeilijk te maken? Op verschillende kerkhoven doet men al aan digitalizatie. Zoals \Autocite{Haekkilae2022} heeft onderzocht, heeft digitalizatie een positieve impact bij bezoekers: 1 deelnemer verteld: 
“We hadden 5 jaar nodig om onze familielid zijn graf te vinden. Het was zeer moeilijk om te terug te vinden, zeker in de winter.” (P4). Na het testen van de applicatie vonden ze het een handige tool.

 Gelijkaardige applicaties worden hier in de omgeving gebruikt. Soms is dit volledig privé, en dus niet voor de bezoeker. Anderzijds is dit volledig publiek en kan de bezoeker gebruik maken van bepaalde tools. Volgens N. Lesdanon (persoonlijk interview, 9 december 2025) gebruikt gemeente Sint-Gillis-Waas een webapplicatie genaamd WebBGP die enkel door de werknemers wordt gebruikt en dus niet direct helpt tijdens een bezoek aan het kerkhof. Echter in gemeente Beveren is er een publieke web-tool, ook wel geoloket genoemd. Dit maakt direct een positieve impact op alle bezoekers, omdat ze deze ofwel vooraf ofwel tijdens het bezoek kunnen raadplegen om zo op een gebruiksvriendelijke manier tot hun gewenste plaats te komen. Ze kunnen hun smartphone gebruiken om naar de website te surfen en het geoloket te raadplegen. Dit wil zeggen dat ze een kaart in beeld krijgen met een route naar hun gewenste graf. Beide tools werden door éénzelfde bedrijf ontwikkeld genaamd Cevi. Geoloketten en applicaties in die aard bestaan er dus zeker al. Toch hebben applicaties nog een nood aan een upgrade. Zoals gereporteerd in \Autocite{Haekkilae2022}, waren de werknemers van de begrafenis zeer positief, toch zijn er nog zorgen over de gebruiksvriendelijkheid van een mobiele applicatie: “Zeker oude mensen gaan het moeilijker hebben met het vinden van grafstenen aangezien ze meestal nog geen smartphone gebruiken.” (P2). Augmented reality bied hier een oplossing op. “Augmented reality (AR) biedt een veelbelovende oplossing voor deze uitdaging door de noodzaak voor gebruikers om hun aandacht te verdelen te minimaliseren, wat het situationeel bewustzijn vergroot door de integratie van virtuele content in de echte wereld.” \autocite{Zhao2025}



%---------- Methodologie ------------------------------------------------------
\section{Methodologie}%
\label{sec:methodologie}
Nu we weten dat navigeren op een kerkhof vaak traag of verwarrend is en zelfs met hedendaagse applicaties nog andere zorgen met zich brengt, is het doel concreet om een Proof of Concept op te stellen met als oplossing Augmented Reality(AR) te gebruiken samen met een geoloket als een mobiele applicatie. Het kerkhof van Sint-Gillis-Waas is relatief klein en heeft nog geen publieke applicatie. Hierop kan een geoloket gemaakt worden met een basis AR-overlay die de route aantoont via de mobiele camera. Er kan eventueel echte data gebruikt worden, want de privé applicatie WebBGP gebruikt deze al. Zo niet kan er een kleine dataset aangemaakt worden met mockdata. Deze toepassing wordt enkel gemaakt voor mobiele Ios apparaten. Er wordt geprogrammeerd in Swift. Hierdoor kan er gebruik gemaakt worden van ARKit voor positionele tracking en alle AR zaken. Wanneer de gebruiker zijn gewenste persoon heeft ingevoerd door bepaalde filters (naam, rijksregisternummer, ...) wordt er een bestemming klaargezet. De gebruiker kan snel switchen tussen de camera met de AR functie, of de kaart met een navigatielijn naar de bestemming. We maken dus degelijk gebruik van Augmented Reality door een effectieve pijl en navigatielijn naar de gewenste grafsteen in de omgeving van de gebruiker te brengen die zichtbaar is door de camera van het mobiele apparaat. Als de gebruiker hierdoor rond de omgeving kijkt is het zeer duidelijk waar hij naartoe moet wandelen, ook al is het kerkhof zeer groot of zijn er slechte weersomstandigheden, wat 1 van de factoren was bij een vermoeiend bezoek. \Autocite{Haekkilae2022}. Dit prototype kan getest worden, hiervoor zijn er een aantal succescriteria waarop we kunnen letten: gebruiker maakt X aantal navigatiefouten, ervaart de gebruiker moeite of stress tijdens gebruik, met gebruik van AR is het X aantal \% sneller dan zonder AR, hoe stabiel is de AR, ... Voor dat we het volledige project uitwerken is het maken van een risicoanalyse zeer cruciaal. Belangrijk bij Augmented Reality zijn bepaalde obstakels, hoogteverschillen, het weer of licht die de tracking kan beïnvloeden. Gebruikers die niet vertrouwd zijn met AR moeten nog steeds gebruik kunnen maken van de app zonder enige problemen. Er wordt geen effectieve integratie met de kerkhofadministratie of gemeente verwerkt, enkel een Proof of Concept die eventueel gebruik kan maken van hun data. Als evaluatiemethode vergelijken we verschillende situaties van de bezoeker: geen gebruik van applicatie, gebruik van applicatie zonder AR, gebruik van applicatie met AR. Bij elke test wordt er een technische evaluatie aan gekoppeld om te zien hoe stabiel en helpvol de Augmented Reality kan zijn.
%---------- Verwachte resultaten ----------------------------------------------
\section{Verwacht resultaat, conclusie}%
\label{sec:verwachte_resultaten}

Hier beschrijf je welke resultaten je verwacht. Als je metingen en simulaties uitvoert, kan je hier al mock-ups maken van de grafieken samen met de verwachte conclusies. Benoem zeker al je assen en de onderdelen van de grafiek die je gaat gebruiken. Dit zorgt ervoor dat je concreet weet welk soort data je moet verzamelen en hoe je die moet meten.

Wat heeft de doelgroep van je onderzoek aan het resultaat? Op welke manier zorgt jouw bachelorproef voor een meerwaarde?

Hier beschrijf je wat je verwacht uit je onderzoek, met de motivatie waarom. Het is \textbf{niet} erg indien uit je onderzoek andere resultaten en conclusies vloeien dan dat je hier beschrijft: het is dan juist interessant om te onderzoeken waarom jouw hypothesen niet overeenkomen met de resultaten.

