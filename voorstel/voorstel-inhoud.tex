%---------- Inleiding ---------------------------------------------------------
\section{Inleiding}%
\label{sec:inleiding}

In dit onderzoek richten we ons op het verbeteren van bestaande navigatie toepassingen met Augmented Reality. Er wordt in vraag gesteld op welke manier bezoekers van kerkhoven op dit moment afzien met de huidige applicaties, en op welke manier Augmented Reality hier een praktische oplossing voor kan bieden. Hoe kan dit op een gebruiksvriendelijke manier toegepast worden in combinatie met een digitale kaart of geoloket zodat elke bezoeker, ongeacht de leeftijd of wat voor soort kerkhof of grafsteen er gezocht wordt, in rust en vrede hun geliefde kan bezoeken.

Op vlak van informatieverzameling wordt er gekeken welke data er beschikbaar is, hoe deze verzameld kan worden en op welke manier deze gepubliceerd en gebruikt kan worden zonder de privacy van de betrokken personen te schenden. Hier is het belangrijk welke data er privé blijft, en welke data er publiek gebruikt mag worden in een mobiele navigatietoepassing.

Wanneer er een Proof of Concept wordt volbracht op een succesvolle manier, kan er gesproken worden van een geslaagd onderzoek. Dit wil zeggen dat er een prototype getest moet worden, en hierop een uitgebreide analyse moet worden gedaan zodat men hieruit definitieve conclusies kan trekken. Op welke manier beïnvloed Augmented Reality nu juist zo'n applicatie? Kan Augmented Reality efficiënt runnen op elk apparaat zonder haperingen of stoornissen te veroorzaken? Bewaard zo'n applicatie de rust die een bezoeker nodig heeft tijdens een meestal emotionele ervaring op een kerkhof? Dit zijn een aantal vragen die we kunnen beantwoorden na uitvoering van de analyse.

%---------- Stand van zaken ---------------------------------------------------

\section{Literatuurstudie}%
\label{sec:literatuurstudie}

% Voor literatuurverwijzingen zijn er twee belangrijke commando's:
% \autocite{KEY} => (Auteur, jaartal) Gebruik dit als de naam van de auteur
%   geen onderdeel is van de zin.
% \textcite{KEY} => Auteur (jaartal)  Gebruik dit als de auteursnaam wel een
%   functie heeft in de zin (bv. ``Uit onderzoek door Doll & Hill (1954) bleek
%   ...'' Als je informatie over bronnen verzamelt in JabRef, zorg er dan voor dat alle nodige info aanwezig is om de bron terug te vinden (zoals uitvoerig besproken in de lessen Research Methods).)

Wanneer men een bezoek brengt aan het kerkhof verwacht je graag geen stress of vermoeidheid. Het vinden van een dierbaar persoon kan vaak langer duren dan verwacht bij een begraafplaats. Dit kan bijvoorbeeld door een groot aantal bezoekers, of een zeer grote of complex gelegen begraafplaats. Hillig Meer is hier een goed voorbeeld van. Een bezoeker vertelde: “Ik stond er gewoon naast. Echt… ik keek er overheen. Het was even verwarrend” \autocite{Meer2025} Hier is het zeer moeilijk om graven te vinden omdat er geen fysieke grafsteen is. Het fysiek zoeken kan dus zeer tijdrovend zijn en is niet altijd even efficiënt. Dit resulteert in een stressvol of ongemakkelijk gevoel, wat niet prettig is bij een bezoek van een geliefde. \Autocite{Haekkilae2022} 

Hierop kan digitalisatie een oplossing bieden. In het dagelijks leven en zeker in 2025, zie je overal digitalisatie. Maar hoe zit dat op kerkhoven? Kan dit op een efficiënte  manier toegepast worden zonder het de bezoekers te moeilijk te maken? Op verschillende kerkhoven doet men al aan digitalisatie. Zoals \Autocite{Haekkilae2022} heeft onderzocht, heeft digitalisatie een positieve impact bij bezoekers: 1 deelnemer verteld: 
“We hadden 5 jaar nodig om onze familielid zijn graf te vinden. Het was zeer moeilijk om te terug te vinden, zeker in de winter.” (P4). Na het testen van de applicatie vonden ze het een handige tool.

 Gelijkaardige applicaties worden hier in de omgeving gebruikt. Soms is dit volledig privé, en dus niet voor de bezoeker. Anderzijds is dit volledig publiek en kan de bezoeker gebruik maken van bepaalde tools. Volgens N. Lesdanon (persoonlijk interview, 9 december 2025) gebruikt gemeente Sint-Gillis-Waas een webapplicatie genaamd WebBGP die enkel door de werknemers wordt gebruikt en dus niet direct helpt tijdens een bezoek aan het kerkhof. Echter in gemeente Beveren is er een publieke webtool, ook wel geoloket genoemd. Dit maakt direct een positieve impact op alle bezoekers, omdat ze deze ofwel vooraf ofwel tijdens het bezoek kunnen raadplegen om zo op een gebruiksvriendelijke manier tot hun gewenste plaats te komen. Ze kunnen hun smartphone gebruiken om naar de website te surfen en het geoloket te raadplegen. Dit wil zeggen dat ze een kaart in beeld krijgen met een route naar hun gewenste graf. Beide tools werden door éénzelfde bedrijf ontwikkeld genaamd Cevi. Geoloketten en applicaties in die aard bestaan er dus zeker al. Toch hebben applicaties nog een nood aan een upgrade. Zoals gereporteerd in \Autocite{Haekkilae2022}, waren de werknemers van de begrafenis zeer positief, toch zijn er nog zorgen over de gebruiksvriendelijkheid van een mobiele applicatie: “Zeker oude mensen gaan het moeilijker hebben met het vinden van grafstenen aangezien ze meestal nog geen smartphone gebruiken.” (P2). Augmented reality bied hier een oplossing op. “Augmented reality (AR) biedt een veelbelovende oplossing voor deze uitdaging door de noodzaak voor gebruikers om hun aandacht te verdelen te minimaliseren, wat het situationeel bewustzijn vergroot door de integratie van virtuele content in de echte wereld.” \autocite{Zhao2025}



%---------- Methodologie ------------------------------------------------------
\section{Methodologie}%
\label{sec:methodologie}
Nu we weten dat navigeren op een kerkhof vaak traag of verwarrend is en zelfs met hedendaagse applicaties nog andere zorgen met zich brengt, is het doel concreet om een Proof of Concept op te stellen met als oplossing Augmented Reality(AR) te gebruiken samen met een geoloket als een mobiele applicatie. Het kerkhof van Sint-Gillis-Waas is relatief klein en heeft nog geen publieke applicatie. Hierop kan een geoloket gemaakt worden met een basis AR-overlay die de route aantoont via de mobiele camera. Er kan eventueel echte data gebruikt worden, want de privé applicatie WebBGP gebruikt deze al. Zo niet kan er een kleine dataset aangemaakt worden met mockdata. Deze toepassing wordt enkel gemaakt voor mobiele Ios apparaten. Er wordt geprogrammeerd in Swift. Hierdoor kan er gebruik gemaakt worden van ARKit voor positionele tracking en alle AR zaken. Wanneer de gebruiker zijn gewenste persoon heeft ingevoerd door bepaalde filters (naam, rijksregisternummer, ...) wordt er een bestemming klaargezet. De gebruiker kan snel switchen tussen de camera met de AR functie, of de kaart met een navigatielijn naar de bestemming. We maken dus degelijk gebruik van Augmented Reality door een effectieve pijl en navigatielijn naar de gewenste grafsteen in de omgeving van de gebruiker te brengen die zichtbaar is door de camera van het mobiele apparaat. Als de gebruiker hierdoor rond de omgeving kijkt is het zeer duidelijk waar hij naartoe moet wandelen, ook al is het kerkhof zeer groot of zijn er slechte weersomstandigheden, wat 1 van de factoren was bij een vermoeiend bezoek. \Autocite{Haekkilae2022}. Dit prototype kan getest worden, hiervoor zijn er een aantal succescriteria waarop we kunnen letten: gebruiker maakt X aantal navigatiefouten, ervaart de gebruiker moeite of stress tijdens gebruik, met gebruik van AR is het X aantal \% sneller dan zonder AR, hoe stabiel is de AR, ... Voor dat we het volledige project uitwerken is het maken van een risicoanalyse zeer cruciaal. Belangrijk bij Augmented Reality zijn bepaalde obstakels, hoogteverschillen, het weer of licht die de tracking kan beïnvloeden. Gebruikers die niet vertrouwd zijn met AR moeten nog steeds gebruik kunnen maken van de app zonder enige problemen. Er wordt geen effectieve integratie met de kerkhofadministratie of gemeente verwerkt, enkel een Proof of Concept die eventueel gebruik kan maken van hun data. Als evaluatiemethode vergelijken we verschillende situaties van de bezoeker: geen gebruik van applicatie, gebruik van applicatie zonder AR, gebruik van applicatie met AR. Bij elke test wordt er een technische evaluatie aan gekoppeld om te zien hoe stabiel en hulpvol de Augmented Reality kan zijn. Op Figuur 1 kan de concrete workflow gezien worden. De volgorde is hier cruciaal. De datums zijn een schatting van hoe lang elk deel zou kunnen innemen. Deze zijn dus niet definitief.

\begin{figure}[H]
    \centering
    \includegraphics[width=0.4\textwidth]{flowchart.png}
    \caption{Flowchart}
\end{figure}
%---------- Verwachte resultaten ----------------------------------------------
\section{Verwacht resultaat, conclusie}%
\label{sec:verwachte_resultaten}

Als alles in goede banen verloopt, kunnen bezoekers van kerkhoven in de toekomst Augmented Reality in mobiele navigatieapplicaties verwachten. Er werd een succesvol prototype getest waardoor de analyses resulteren in een grote groei van efficiëntie tijdens het navigeren van een kerkhof binnen verschillende omstandigheden. We kunnen dus concluderen dat gebruik van ARKit in mobiele Ios apparaten een gepast antwoord is op de hoofdonderzoeksvraag. Hierdoor kan de Proof of Concept dienen als eerste bouwsteen om een mobiele applicatie realiteit te maken. Dit kan beginnen in bijvoorbeeld gemeente Sint-Gillis-Waas en omstreken, omdat zij nog geen enkele vorm van publieke mobiele applicatie tijdens een kerkhofbezoek bezitten.
